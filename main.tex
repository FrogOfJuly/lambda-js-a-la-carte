%%
%% This is file `sample-manuscript.tex',
%% generated with the docstrip utility.
%%
%% The original source files were:
%%
%% samples.dtx  (with options: `all,proceedings,bibtex,manuscript')
%% 
%% IMPORTANT NOTICE:
%% 
%% For the copyright see the source file.
%% 
%% Any modified versions of this file must be renamed
%% with new filenames distinct from sample-manuscript.tex.
%% 
%% For distribution of the original source see the terms
%% for copying and modification in the file samples.dtx.
%% 
%% This generated file may be distributed as long as the
%% original source files, as listed above, are part of the
%% same distribution. (The sources need not necessarily be
%% in the same archive or directory.)
%%
%%
%% Commands for TeXCount
%TC:macro \cite [option:text,text]
%TC:macro \citep [option:text,text]
%TC:macro \citet [option:text,text]
%TC:envir table 0 1
%TC:envir table* 0 1
%TC:envir tabular [ignore] word
%TC:envir displaymath 0 word
%TC:envir math 0 word
%TC:envir comment 0 0
%%
%% The first command in your LaTeX source must be the \documentclass
%% command.
%%
%% For submission and review of your manuscript please change the
%% command to \documentclass[manuscript, screen, review]{acmart}.
%%
%% When submitting camera ready or to TAPS, please change the command
%% to \documentclass[sigconf]{acmart} or whichever template is required
%% for your publication.
%%
%%
\documentclass[sigplan,nonacm]{acmart}

%%
%% \BibTeX command to typeset BibTeX logo in the docs
\AtBeginDocument{%
  \providecommand\BibTeX{{%
    Bib\TeX}}}

%% Rights management information.  This information is sent to you
%% when you complete the rights form.  These commands have SAMPLE
%% values in them; it is your responsibility as an author to replace
%% the commands and values with those provided to you when you
%% complete the rights form.
\setcopyright{acmlicensed}
\copyrightyear{2025}
\acmYear{2026}
\acmDOI{XXXXXXX.XXXXXXX}
%% These commands are for a PROCEEDINGS abstract or paper.
\acmConference[Conference acronym 'XX]{Make sure to enter the correct
  conference title from your rights confirmation email}{June 03--05,
  2018}{Woodstock, NY}
%%
%%  Uncomment \acmBooktitle if the title of the proceedings is different
%%  from ``Proceedings of ...''!
%%
%%\acmBooktitle{Woodstock '18: ACM Symposium on Neural Gaze Detection,
%%  June 03--05, 2018, Woodstock, NY}
\acmISBN{978-1-4503-XXXX-X/2018/06}
%%
%% Submission ID.
%% Use this when submitting an article to a sponsored event. You'll
%% receive a unique submission ID from the organizers
%% of the event, and this ID should be used as the parameter to this command.
%%\acmSubmissionID{123-A56-BU3}

%%
%% For managing citations, it is recommended to use bibliography
%% files in BibTeX format.
%%
%% You can then either use BibTeX with the ACM-Reference-Format style,
%% or BibLaTeX with the acmnumeric or acmauthoryear sytles, that include
%% support for advanced citation of software artefact from the
%% biblatex-software package, also separately available on CTAN.
%%
%% Look at the sample-*-biblatex.tex files for templates showcasing
%% the biblatex styles.
%%

%%
%% The majority of ACM publications use numbered citations and
%% references.  The command \citestyle{authoryear} switches to the
%% "author year" style.
%%
%% If you are preparing content for an event
%% sponsored by ACM SIGGRAPH, you must use the "author year" style of
%% citations and references.
%% Uncommenting
%% the next command will enable that style.
% \citestyle{acmauthoryear}

\usepackage{listings}
\usepackage{xcolor}

\definecolor{codegreen}{rgb}{0,0.6,0}
\definecolor{codegray}{rgb}{0.5,0.5,0.5}
\definecolor{codepurple}{rgb}{0.58,0,0.82}
\definecolor{backcolour}{rgb}{0.95,0.95,0.92}

\lstdefinestyle{mystyle}{
    % backgroundcolor=\color{backcolour},   
    commentstyle=\color{codegreen},
    keywordstyle=\color{magenta},
    numberstyle=\tiny\color{codegray},
    stringstyle=\color{codepurple},
    basicstyle=\ttfamily\footnotesize,
    breakatwhitespace=false,         
    breaklines=true,                 
    captionpos=b,                    
    keepspaces=true,                 
    numbers=left,                    
    numbersep=5pt,                  
    showspaces=false,                
    showstringspaces=false,
    showtabs=false,                  
    tabsize=2
}

\lstset{style=mystyle}

\usepackage{todonotes}
\setlength {\marginparwidth }{2cm}


%%
%% end of the preamble, start of the body of the document source.
\begin{document}

%%
%% The "title" command has an optional parameter,
%% allowing the author to define a "short title" to be used in page headers.
\title{JavaScript à la Carte}
`'
%%
%% The "author" command and its associated commands are used to define
%% the authors and their affiliations.
%% Of note is the shared affiliation of the first two authors, and the
%% "authornote" and "authornotemark" commands
%% used to denote shared contribution to the research.

\author{Kirill Golubev}
\email{kirill.golubev@utu.fi}
\orcid{0009-0002-2709-5241}
\affiliation{%
  \institution{University of Turku}
  \city{Turku}
  \country{Finland}
}
% \authornote{
%   Supervisor: Jaakko Järvi, jaakko.jarvi@utu.fi, University of Turku, Turku, Finland, orcid: 0000-0002-3418-7366\\
%   Supervisor: Mikhail Barash, mikhail.barash@uib.no, University of Bergen, Bergen, Norway, orcid: 0000-0002-7067-2588
% }


%%
%% By default, the full list of authors will be used in the page
%% headers. Often, this list is too long, and will overlap
%% other information printed in the page headers. This command allows
%% the author to define a more concise list
%% of authors' names for this purpose.
\renewcommand{\shortauthors}{Golubev}

%%
%% The code below is generated by the tool at http://dl.acm.org/ccs.cfm.
%% Please copy and paste the code instead of the example below.
%%
\begin{CCSXML}
<ccs2012>
 <concept>
<concept_id>10011007.10010940.10010992.10010998.10010999</concept_id>
<concept_desc>Software and its engineering~Software verification</concept_desc>
<concept_significance>500</concept_significance>
</concept>
</ccs2012>
\end{CCSXML}

\ccsdesc[500]{Software and its engineering~Software verification}

%%
%% Keywords. The author(s) should pick words that accurately describe
%% the work being presented. Separate the keywords with commas.
\keywords{JavaScript, formal methods, verification, program equivalence}

% \received{20 February 2007}
% \received[revised]{12 March 2009}
% \received[accepted]{5 June 2009}

\maketitle

\section{Introduction}

\todo[inline]{Half a page for overview and explaining motivation}

\section{à-la-carte-ness}

\todo[inline]{Why Rocq\cite{the_coq_development_team_2024_14542673}, coq a la carte as a foundation, why can't use directly, imporovements, what I want to do}

\todo[inline]{Main citation: Coq à la Carte\cite{forster2020coq}}

\section{Targets for formalization}

There is yet to be a significant test of modularity for mainstream programming languages formalization.

We propose a case study that would evaluate existing approaches for modular Rocq developments. 
The aim of the study is to find and address, when possible, their shortcomings, improve tooling and enable them to support practical proofs. 
We choose JavaScript as a target language for mechanisation.

There are several\cite{guha2010essence}\cite{bodin2014trusted} developments that attempt to formalize and reason about JavaScript, however non of them is easy to extend for with new features.
Being one of the most used language, JavaScript provides a fertile ground for evaluating existing approaches for modular reasoning. 
Lack of sophisticated type system streamlines the encoding and makes it possible to gradually prove language properties, while keeping the formalization open to extension.
Extensive specification\cite{ECMA} in natural language is also a very welcome addition.

To the best of our knowledge no modular technique from above was ever used for mechanisation of a mainstream language.

The following theorems seem to be good candidates for the study: 

\begin{itemize}
  \item preservation of local closeness w.r.t small step semantics

\begin{lstlisting}[numbers=none]
forall c e c' e', lc e 
               -> step c e c' e' 
               -> lc e'. 
\end{lstlisting}

\item progress theorem 
\begin{lstlisting}[numbers=none]
forall c e, lc e 
         -> isValue e 
         \/ isError e
         \/ (exists c' e', step c e c' e').
\end{lstlisting}
\end{itemize}

\todo[inline]{Find citation for dialects and t39 proposal}

Moreover, JavaScript has several frameworks\cite{React} and dialects\cite{}  that enable different styles of programming. 
Ability to reuse proofs about core language for dialects would be a nice showcase of modularity. 
The t39 proposal process\cite{} is transparent and permits mechanization of ongoing specification of nightly features before they are adopted to core language.

\section{Discussion}

There are other solutions to increase modularity of proofs. 

\todo[inline]{Proof modularity papaers}

\begin{enumerate}
  \item Family Polymorphism\cite{jin2023extensible}
  \todo[inline]{Rocq plugin for type family polymorphism}
  \item Program Logics à la Carte\cite{vistrup2025program}
  \todo[inline]{Coinduction with ITrees.}
  \item Interpreters à la Carte\cite{van2022intrinsically}
  \todo[inline]{Containers as functors for fixpoints}
\end{enumerate}

\todo[inline]{Argue about that indirect encoding is too taxing.}

\todo[inline]{It's interesting to look into the possibility of gradually encoding calculus of inductive constructions in a modular fashion.}

Proof modularity comes with the cost of departing from the usual way of reasoning about inductive types. 
Even in case of Coq à la Carte departure is not quite dramatic, but still requires to rethink how one approaches proofs.

The ideal solution would be to have a correspondence between modular and inductive proofs. 
There is an existing work\cite{tabareau2021marriage}\cite{cohen2024trocq} that could enable that proofs transfer between "equivalent" datatypes. 

\todo[inline]{Talk about how they achieve that and is it possible to leverage that for functor representation of chosen datatype.}

\todo[inline]{Is it possible to use containers as a meta language, while presreving ability to do actual reasoning with inductive types in Rocq?}


\bibliographystyle{ACM-Reference-Format}
\bibliography{lib}

\appendix


\end{document}
